\documentclass[]{article}

%opening
\title{The Collatz Conjecture}
\author{Joshua B. McBride}

\begin{document}

\maketitle

\begin{abstract}
The Collatz conjecture is a conjecture in mathematics named after Lothar Collatz. It concerns a sequence defined as follows: start with any positive integer n. Then each term is obtained from the previous term as follows: if the previous term is even, the next term is one half the previous term. Otherwise, the next term is 3 times the previous term plus 1. The conjecture is that no matter what value of n, the sequence will always reach 1.
\end{abstract}

\section{Introduction}
There have been 
\section{Evens and Odds}

\section{Algorithmic Functions}

\section{{Possible Set Movements}
\end{document}
